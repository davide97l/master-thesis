% !TeX root = ../thuthesis-example.tex

\begin{resume}

  %\section*{Resume}
  % Please note the empty line to ensure the correct format for this section.
    Davide Liu was born on 27th August 1997 in Padova, Veneto, Italy. 
    
    He began his bachelor’s study in the Department of Computer Science, Padova University in September 2016, majoring in computer science, and got a Bachelor of Science degreeth in July 2019.
    
    He began his master’s study in the Department of Computer Science, Tsinghua University in September 2019, and is expected to get a Master of Science degree in Computer Science in July 2021.
    
    %She/He has started to pursue a doctor’s degree in Physics in the School/Department of XXX, Tsinghua University since September 2014. During this period, she/he has made academic achievements as follows.
    


  %\section*{Academic Achievements}

  % Please delete the following sections if they are not applicable.
  
  %\subsection*{\textbf{Monograph}}
  %\begin{achievements}
  %  \item Author. Title. Publishing Group. In standard referencing format
  %\end{achievements}
  %\subsection*{\textbf{Journal article}}
  % Please put your publications here
  %\begin{achievements}
  %  \item Yang Y, Ren T L, Zhang L T, et al. Miniature microphone with silicon-based ferroelectric thin films[J]. Integrated Ferroelectrics, 2003, 52:229-235.
  %  \item 杨轶, 张宁欣, 任天令, 等. 硅基铁电微声学器件中薄膜残余应力的研究[J]. 中国机械工程, 2005, 16(14):1289-1291.
   % \item 杨轶, 张宁欣, 任天令, 等. 集成铁电器件中的关键工艺研究[J]. 仪器仪表学报, 2003, 24(S4):192-193.
%    \item Yang Y, Ren T L, Zhu Y P, et al. PMUTs for handwriting recognition. In press[J]. (已被Integrated Ferroelectrics录用)
 % \end{achievements}


%  \subsection*{\textbf{Patent}}

%  \begin{achievements}
%    \item 任天令, 杨轶, 朱一平, 等. 硅基铁电微声学传感器畴极化区域控制和电极连接的方法: 中国, CN1602118A[P]. 2005-03-30.
%    \item Ren T L, Yang Y, Zhu Y P, et al. Piezoelectric micro acoustic sensor based on ferroelectric materials: USA, No.11/215, 102[P]. (American Patent number...)
%  \end{achievements}

\end{resume}
