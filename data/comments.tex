% !TeX root = ../thuthesis-example.tex

\chapter{COMMENTS FROM THESIS SUPERVISOR}

深度学习模型的对抗攻击与防御是当前研究的热点,是深度学习模型在应用过程中需要重点关注的问题。目前在图像识别等任务上,相关研究进展较多。但对于使用了深度神经网络的深度强化学习,这方面的研究相对较少。此外,对抗攻击与防御是矛与盾的关系,一方面的进展会促进另一方面的提升,因此,如何进行充分的评测是其中要解决的重要问题。 为此,该论文针对一些常见的深度强化学习算法的对抗鲁棒性进行评测,通过多个决策任务下的实验,展示它们在恶意策略攻击下的迁移性,同时,也对基于图像的对抗攻击进行了实验测试。 该论文的选题具有实际应用价值。作为一个实验验证的论文,相关实验在多个任务场景下开展,工作量充足,结论合理。论文写作符合规范。满足硕士论文的要求。
